% !TeX document-id = {f19fb972-db1f-447e-9d78-531139c30778}
% !BIB program = biber

%\documentclass[handout]{beamer}
\documentclass[compress]{beamer}
\usepackage[T1]{fontenc}
\usetheme[block=fill,subsectionpage=progressbar,sectionpage=progressbar]{metropolis} 
\usepackage{graphicx}

\usepackage{wasysym}
\usepackage{etoolbox}
\usepackage[utf8]{inputenc}

\usepackage{threeparttable}
\usepackage{subcaption}

\usepackage{tikz-qtree}
\setbeamercovered{still covered={\opaqueness<1->{5}},again covered={\opaqueness<1->{100}}}


\usepackage{listings}

\lstset{
	basicstyle=\scriptsize\ttfamily,
	columns=flexible,
	breaklines=true,
	numbers=left,
	%stepsize=1,
	numberstyle=\tiny,
	backgroundcolor=\color[rgb]{0.85,0.90,1}
}



\lstnewenvironment{lstlistingoutput}{\lstset{basicstyle=\footnotesize\ttfamily,
		columns=flexible,
		breaklines=true,
		numbers=left,
		%stepsize=1,
		numberstyle=\tiny,
		backgroundcolor=\color[rgb]{.7,.7,.7}}}{}


\lstnewenvironment{lstlistingoutputtiny}{\lstset{basicstyle=\tiny\ttfamily,
		columns=flexible,
		breaklines=true,
		numbers=left,
		%stepsize=1,
		numberstyle=\tiny,
		backgroundcolor=\color[rgb]{.7,.7,.7}}}{}



\usepackage[american]{babel}
\usepackage{csquotes}
\usepackage[style=apa, backend = biber]{biblatex}
\DeclareLanguageMapping{american}{american-UoN}
\addbibresource{../references.bib}
\renewcommand*{\bibfont}{\tiny}

\usepackage{tikz}
\usetikzlibrary{shapes,arrows,matrix}
\usepackage{multicol}

\usepackage{subcaption}

\usepackage{booktabs}
\usepackage{graphicx}



\makeatletter
\setbeamertemplate{headline}{%
	\begin{beamercolorbox}[colsep=1.5pt]{upper separation line head}
	\end{beamercolorbox}
	\begin{beamercolorbox}{section in head/foot}
		\vskip2pt\insertnavigation{\paperwidth}\vskip2pt
	\end{beamercolorbox}%
	\begin{beamercolorbox}[colsep=1.5pt]{lower separation line head}
	\end{beamercolorbox}
}
\makeatother



\setbeamercolor{section in head/foot}{fg=normal text.bg, bg=structure.fg}



\newcommand{\question}[1]{
	\begin{frame}[plain]
		\begin{columns}
			\column{.3\textwidth}
			\makebox[\columnwidth]{
				\includegraphics[width=\columnwidth,height=\paperheight,keepaspectratio]{../pictures/mannetje.png}}
			\column{.7\textwidth}
			\large
			\textcolor{orange}{\textbf{\emph{#1}}}
		\end{columns}
\end{frame}}


\title[Teach-the-teacher: Python]{\textbf{Teach-the-teacher: Python} 
\\Day 4: » Advanced NLP \& Regular Expressions « }
\author[Anne Kroon]{Anne Kroon\\ \footnotesize{a.c.kroon@uva.nl}}
\date{December 4, 2023}
\institute[UvA CW]{UvA RM Communication Science}


\begin{document}
	
	\begin{frame}{}
		\titlepage
	\end{frame}
	
	\begin{frame}{Today}
		\tableofcontents
	\end{frame}


\section{Advanced NLP}

\subsection{Parsing sentences}
\begin{frame}{NLP: What and why?}
\begin{block}{Why parse sentences?}
\begin{itemize}
	\item To find out what grammatical function words have
	\item and to get closer to the meaning.
\end{itemize}
\end{block}
\end{frame}

\begin{frame}[fragile]{Parsing a sentence using NLTK}
Tokenize a sentence, and ``tag'' the tokenized sentence:
\begin{lstlisting}
tokens = nltk.word_tokenize(sentence)
tagged = nltk.pos_tag(tokens)
print (tagged[0:6])
\end{lstlisting}
gives you the following:
\begin{lstlisting}
[('At', 'IN'), ('eight', 'CD'), ("o'clock", 'JJ'), ('on', 'IN'),
('Thursday', 'NNP'), ('morning', 'NN')]
\end{lstlisting}

\onslide<2->{
And you could get the word type of "morning" with \texttt{tagged[5][1]}!
}

\end{frame}


\begin{frame}[fragile]{Named Entity Recognition with spacy}
Terminal:

\begin{lstlisting}
sudo pip3 install spacy
sudo python3 -m spacy download nl    # or en, de, fr ....
\end{lstlisting}

Python:

\begin{lstlisting}
import spacy
nlp = spacy.load('nl')
doc = nlp('Een 38-jarige vrouw uit Zeist en twee mannen moeten 24 maanden de cel in voor de gecoordineerde oplichting van Rabobank-klanten.')
for ent in doc.ents:
    print(ent.text,ent.label_)
\end{lstlisting}

returns:

\begin{lstlisting}
Zeist LOC
Rabobank ORG
\end{lstlisting}  

\end{frame}



\begin{frame}{More NLP}
\url{http://nlp.stanford.edu}
\url{http://spacy.io}
\url{http://nltk.org}
\url{https://www.clips.uantwerpen.be/pattern}
\end{frame}



\begin{frame}{Main takeaway}

\begin{itemize}
%	\item It matters how you transform your text into numbers (``vectorization'').
\item Preprocessing matters, be able to make informed choices.
\item Keep this in mind when moving to Machine Learning. 
\end{itemize}
\end{frame}


\section[Regular expressions]{ACA using regular expressions}

\begin{frame}
Automated content analysis using regular expressions
\end{frame}


\subsection{What is a regexp?}
\begin{frame}{Regular Expressions: What and why?}
\begin{block}{What is a regexp?}
\begin{itemize}
\item<1-> a \emph{very} widespread way to describe patterns in strings
\item<2-> Think of wildcards like {\tt{*}} or operators like {\tt{OR}}, {\tt{AND}} or {\tt{NOT}} in search strings: a regexp does the same, but is \emph{much} more powerful
\item<3-> You can use them in many editors (!), in the Terminal, in STATA \ldots and in Python
\end{itemize}
\end{block}
\end{frame}

\begin{frame}{An example}
\begin{block}{Regex example}
\begin{itemize}
\item Let's say we wanted to remove everything but words from a tweet
\item We could do so by calling the \texttt{.replace()} method
\item We could do this with a regular expression as well: \\
{\tt{ \lbrack \^{}a-zA-Z\rbrack}} would match anything that is not a letter
\end{itemize}
\end{block}
\end{frame}

\begin{frame}{Basic regexp elements}
\begin{block}{Alternatives}
\begin{description}
\item[{\tt{\lbrack TtFf\rbrack}}] matches either T or t or F or f
\item[{\tt{Twitter|Facebook}}] matches either Twitter or Facebook
\item[{\tt{.}}] matches any character
\end{description}
\end{block}
\begin{block}{Repetition}<2->
\begin{description}
\item[{\tt{*}}] the expression before occurs 0 or more times
\item[{\tt{+}}] the expression before occurs 1 or more times
\end{description}
\end{block}
\end{frame}

\begin{frame}{regexp quizz}
\begin{block}{Which words would be matched?}
\tt
\begin{enumerate}
\item<1-> \lbrack Pp\rbrack ython
\item<2-> \lbrack A-Z\rbrack +
\item<3-> RT ?:? @\lbrack a-zA-Z0-9\rbrack *
\end{enumerate}
\end{block}
\end{frame}

\begin{frame}{What else is possible?}
See the table in the book!
\end{frame}

\subsection{Using a regexp in Python}
\begin{frame}{How to use regular expressions in Python}
\begin{block}{The module \texttt{re}*}
\begin{description}
\item<1->[{\tt{re.findall("\lbrack Tt\rbrack witter|\lbrack Ff\rbrack acebook",testo)}}] returns a list with all occurances of Twitter or Facebook in the string called {\tt{testo}}
\item<1->[{\tt{re.findall("\lbrack 0-9\rbrack +\lbrack a-zA-Z\rbrack +",testo)}}] returns a list with all words that start with one or more numbers followed by one or more letters in the string called {\tt{testo}}
\item<2->[{\tt{re.sub("\lbrack Tt\rbrack witter|\lbrack Ff\rbrack acebook","a social medium",testo)}}] returns a string in which all all occurances of Twitter or Facebook are replaced by "a social medium"
\end{description}
\end{block}

\tiny{Use the less-known but more powerful module \texttt{regex} instead to support all dialects used in the book}
\end{frame}


\begin{frame}[fragile]{How to use regular expressions in Python}
\begin{block}{The module re}
\begin{description}
\item<1->[{\tt{re.match(" +(\lbrack 0-9\rbrack +) of (\lbrack 0-9\rbrack +) points",line)}}] returns  \texttt{None} unless it \emph{exactly} matches the string \texttt{line}. If it does, you can access the part between () with the \texttt{.group()} method.
\end{description}
\end{block}

Example:
\begin{lstlisting}
line="             2 of 25 points"
result=re.match(" +([0-9]+) of ([0-9]+) points",line)
if result:
    print (f"Your points: {}result.group(1)}, Maximum points: {result.group(2)})
\end{lstlisting}
Your points: 2 Maximum points: 25
\end{frame}



\begin{frame}{Possible applications}
\begin{block}{Data preprocessing}
\begin{itemize}
\item Remove unwanted characters, words, \ldots
\item Identify \emph{meaningful} bits of text: usernames, headlines, where an article starts, \ldots
\item filter (distinguish relevant from irrelevant cases)
\end{itemize}
\end{block}
\end{frame}


\begin{frame}{Possible applications}
\begin{block}{Data analysis: Automated coding}
\begin{itemize}
\item Actors
\item Brands
\item links or other markers that follow a regular pattern
\item Numbers (!)
\end{itemize}
\end{block}
\end{frame}

\begin{frame}[fragile,plain]{Example 1: Counting actors}
\begin{lstlisting}
import re, csv
from glob import glob
count1_list=[]
count2_list=[]
filename_list = glob("/home/damian/articles/*.txt")

for fn in filename_list:
    with open(fn) as fi:
        artikel = fi.read()
        artikel = artikel.replace('\n',' ')

    count1 = len(re.findall('Israel.*(minister|politician.*|[Aa]uthorit)',artikel))
    count2 = len(re.findall('[Pp]alest',artikel))

    count1_list.append(count1)
    count2_list.append(count2)

output=zip(filename_list,count1_list, count2_list)
with open("results.csv", mode='w',encoding="utf-8") as fo:
    writer = csv.writer(fo)
    writer.writerows(output)
\end{lstlisting}
\end{frame}




\begin{frame}[fragile]{Example 2: Which number has this Lexis Nexis article?}
\begin{lstlisting}
                        All Rights Reserved

                  2 of 200 DOCUMENTS

            De Telegraaf

         21 maart 2014 vrijdag

Brussel bereikt akkoord  aanpak probleembanken;
ECB krijgt meer in melk te brokkelen

SECTION: Finance; Blz. 24
LENGTH: 660 woorden

BRUSSEL   Europa heeft gisteren op de valreep een akkoord bereikt 
over een saneringsfonds voor banken. Daarmee staat de laatste
\end{lstlisting}

\end{frame}

\begin{frame}[fragile]{Example 2: Check the number of a lexis nexis article}
\begin{lstlisting}
                        All Rights Reserved

                  2 of 200 DOCUMENTS

            De Telegraaf

       21 maart 2014 vrijdag

Brussel bereikt akkoord  aanpak probleembanken;
ECB krijgt meer in melk te brokkelen

SECTION: Finance; Blz. 24
LENGTH: 660 woorden

BRUSSEL   Europa heeft gisteren op de valreep een akkoord bereikt 
over een saneringsfonds voor banken. Daarmee staat de laatste
\end{lstlisting}

\begin{lstlisting}
for line in tekst:
matchObj=re.match(r" +([0-9]+) of ([0-9]+) DOCUMENTS",line)
if matchObj:
    numberofarticle= int(matchObj.group(1))
    totalnumberofarticles= int(matchObj.group(2))
\end{lstlisting}
\end{frame}


\begin{frame}{Practice yourself!}
Let's take some time to write some regular expressions.
Write a script that
\begin{itemize}
\item extracts URLS form a list of strings
\item removes everything that is not a letter or number from a list of strings
\end{itemize}
(first develop it for a single string, then scale up)

More tips:
\huge{\url{http://www.pyregex.com/}}
\end{frame}



%\begin{frame}[plain]
%	\printbibliography
%\end{frame}



\end{document}



